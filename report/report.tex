\documentclass[sigconf,screen,nonacm]{acmart}

\title[BulletID]{BulletID: ePassport Possession Attestation Using Bulletproofs}

\author{Yannick Abouem}
\affiliation{
  \institution{Queen's University}
  \department{School of Computing}
  \city{Kingston}
  \state{ON}
  \country{Canada}
}
\email{24hd7@queensu.ca}

\setcopyright{none}

\begin{abstract}
  In the digital age, it has become more common the need to provide IDs to a website
  in order to access some service. However, doing so is risky as you can not fully
  trust the website operator with such information. To solve this problem, we utilized
  zero-knowledge proofs to construct proofs using the information in the
  machine-readable zone of a passport. Zero-knowledge proofs are a cryptographically
  secure way of constructing proofs of knowledge of some secret piece of information.
  Using the Bulletproofs protocol, we constructed a program that construct a proof
  from the information in a passport and verifies said proof.
\end{abstract}

\begin{document}
  \maketitle

  \section{Introduction}

  With the advent of the digital age, it has become more common the need for a user
  to prove their legal identity over the internet. This tasks often involves filling
  a form containing your own personal information, such as name, date of birth, sex
  etc. and, in occasions, providing a picture or a scan of one of your IDs. Modern
  cryptographic techniques will protect your information in transit and when stored,
  but it would be of little help if the other party were to misuse, either by malice
  or by incompetence, your own information.

  To illustrate this problem, assume a private company $P$ that provides electricity
  to a city. This company has a simple sign-up page to become a costumer, but in
  order to complete the form a prospective costumer has to provide a picture of one
  of their IDs to prove that they are actually a person. Using the current techniques
  the user will have to send said picture over the internet, which will then be
  reviewed by the company $P$. However, the user has no longer control on what
  happens to that picture. Let's consider the case where $P$, due to cost-cutting
  measures, has poor cybersecurity practices and a malicious entity is able to take
  control of their systems. If, let's say, the ID pictures sent to $P$ were not to
  be properly stored with encryption, the attacker will be able to steal the ID
  pictures and use them for their nefarious purposes. This would cause a massive
  data breach that would affect all of $P$'s costumers.

  There exists many ways to improve upon this scenario. For example, $P$ might only
  store the ID pictures for no longer than they need them for. We decided to take a
  different approach from most of the solutions and explored the use of zero-knowledge
  proofs to create proofs of ID possession, specifically passports, to solve
  this problem. As we will see, we created a system to convert the machine-readable
  zone information of a passport into a proof of knowledge of this information
  using the inner-product proofs offered by Bulletproofs.

  \section{Background}

  In this section we will provide background on electronic IDs (eIDs), zero-knowledge
  proofs (ZKP) and Bulletproofs.

  \subsection{Electronic ID}

  To enhance the identity verification process, many countries issue identity documents
  (IDs) with an embedded RFID chip \cite{wiki1}. This chip contains the same
  information as displayed on the physical card or, it may, in addition, store
  biometric information of the holder such as fingerprint and facial recognition
  data \cite{wiki1}.

  EIDs are commonly issued in the form of national IDs but, the most common form of
  eIDs are ePassports. According to the International Civil Aviation Organization
  (ICAO)[\citeyear{ICAO1}], ``[t]here are more than 140 States and non-state entities
  (e.g. United Nations, European Union) currently issuing ePassports, and over 1
  billion ePassports in circulation''. EPassports are a form of electronic Machine
  Readable Travel Documents (eMRTD) and are standardized by ICAO under the Doc 9303.
  \cite{ICAO2, wiki2}. Each passport possesses a machine-readable zone (MRZ) located
  in the passport data page, which is mandatory for all ICAO members to include
  \cite{ICAO2}. This MRZ contains all information about the passport and the
  individual who possesses it. This data is formatted, according to the ICAO
  [\citeyear{ICAO2}], in the following manner:

  On the first line:
  \begin{enumerate}
    \item Document code. One character \emph{P} followed by a character denoting the
      passport type or a ``filler'' character \emph{<}
    \item Issuing state or organization. The three-letter ISO code of the state or
      organization issuing the passport
    \item Name. The name of the individual who owns the passport of length at most 39
      characters. Last name and given names might be separated by ``filler''
      characters. Extra unused characters will be filled with \emph{<}
  \end{enumerate}

  On the second line:
  \begin{enumerate}
    \item Passport number. Nine characters long passport number
    \item Check digit. Check digit of the passport number
    \item Nationality. Three-letter ISO code of the holder's nationality
    \item Date of birth. Date of birth of the holder of length 6
    \item Check digit. Check digit for the date of birth
    \item Sex. Single character denoting the sex of the holder. Either one of M, F or
      <
    \item Date of expiry. Six digit expiriy date of the passport
    \item Check digit. Check digit for the date of expiry
    \item Personal number or other data element. This section has 14 characters in
      length and can be used by a country as they see fit. If unused replaced with
      filler characters.
    \item Check digit. Check digit for the previous section. If previous section is
      empty could be 0 or filler character
    \item Composite check digit. Check digit for all check digits in the second line
  \end{enumerate}

  We will be using the data contained in the MRZ in this project as the information
  we desire to create a zero-knowledge proof of.

  \subsection{A Note About Notation}

  In this report we will use the same notation as used by Bunz et al.
  [\citeyear{Bunz17}] in the original Bulletproofs paper. In particular, we will 
  denote vectors in bold, i.e. $\mathbf{a} \in \mathbb{F}^n$ is a vector with elements
  in $\mathbb{F}$, and the inner product notation
  $c = \langle \mathbf{a}, \mathbf{b} \rangle$. We will also denote cyclic groups of
  prime order $p$ with $\mathbb{G}$ and the ring of integer modulo $p$ as
  $\mathbb{Z}_p$.

  \subsection{Zero-Knowledge Proofs}

  Zero-knowledge Proofs (ZKP) are a cryptographic technique to prove the knowledge of
  a secret piece of information without revealing this information to the verifier
  of the proof. This is particularly useful if we do not fully trust the verifier
  with information such as personal identifiable information (PII) or IDs.
  A ZKP protocol has, usually, two entities, a Prover, who constructs a proof of knowledge
  over some secret information, and a Verifier that must be convinced by the Prover
  that their proof is correct.

  There are three properties a ZKP protocol possesses:
  \begin{enumerate}
    \item \textbf{Correctness}, if an honest verifier is able to convince a Prover
      that they know the secret information \cite{Gibson22}
    \item \textbf{Zero-knowledge}, no secrete information is revealed to the Verifier
      other than the necessary information \cite{Gibson22}
    \item \textbf{Soundness}, if the verification of the proof correlates to proving
      knowledge of the secret information \cite{Gibson22}
  \end{enumerate}
  Protocols rely on one-way functions or circuits to create the proof to ensure that
  it is not computationally feasible for a dishonest Prover or Verifier to violate
  any of the properties.

  There exists many ZKP protocols in the literature. Among these the most popular and
  well-developed are zk-SNARK, zk-STARKS and Bulletproofs. In the following paragraphs
  we will briefly describe the first two protocols and in the next subsection
  we will explore more in depth Bulletproofs and how does the protocol work.

  \subsubsection{zk-SNARK}

  Short for zero-knowledge succinct non-interactive argument of knowledge, zk-SNARK
  is an argument of a NP statement (such as a computation) that is verifiable with
  a small computational complexity \cite{cryptoeprint:2012/095}. This is achieved
  through the use of collision-resistant hash functions and probabilistically-checked
  proofs\cite{cryptoeprint:2012/095}.

  \subsection{Bulletproofs}

  Bulletproofs is a non-interactive zero-knowledge protocol that possesses short proofs
  and does not require a trusted setup, which distinguish it from both zk-SNARKS, which
  require a trusted setup, and zk-STARKS, which proofs are larger but require
  no trusted setup \cite{Bunz17}. Before explaining how Bulletproofs work it is necessary
  to talk about Pedersen commitments.

  \subsubsection{Pedersen Commits}

  A cryptographic commitment scheme is a pair of algorithms $(\mathrm{Setup},\mathrm{Com})$
  such that the algorithm $\mathrm{Setup}$ generates the public parameters for the scheme
  and the algorithm $\mathrm{Com}$ defines a function that given a message and a random value produces a
  commitment to that message \cite{Bunz17}. A commitment scheme must abide by two properties:
  \begin{itemize}
    \item Hiding, a commitment does not reveal the message \cite{Gibson22}
    \item Binding, there is no message $m'$ such that $\mathrm{Com}(m) = \mathrm{Com}(m')$ \cite{Gibson22, Bunz17}
  \end{itemize}
  An example of a commitment scheme is cryptographic hash functions since they possess the
  binding property and it is also possible to obtain the hiding property through the use of
  a random value in the input. However, Bulletproofs require a third property for its commitments,
  Homomorphism.

  A homomorphic commitment is a commitment such that the equation
  \begin{equation}
    \mathrm{Com}(m_1, r_1) + \mathrm{Com}(m_2,r_2) = \mathrm{Com}(m_1 + m_2, r_1 + r_2)
  \end{equation}
  is true \cite{Bunz17}. One such commitment scheme is the Pedersen commitment scheme.

  A vector Pedersen commitment $C$ is a point on an elliptic curve such that:
  \begin{equation}
    C = rH + \sum_{i=1}^n v_i G_i
  \end{equation}
  where $\mathbf{v}$ is a secret vector for which we are building the commitment of,
  $r$ is a randomly generated scalar, $\mathbf{G}$ is a generator vector of the
  elliptic curve previously agreed upon by both parties, and $H$ is a point on the
  curve for which the discrete logarithm $q$ is not known by anyone and $H = q\mathbf{G}$
  \cite{Gibson22}.

  \subsubsection{The Inner-Product Argument}

  The core component of Bulletproofs is the inner-product argument, which is an
  argument of knowledge that proves that the Prover knows the secret vectors in
  the Pedersen commitment that satisfy a specific inner product relation
  \cite{Bunz17}. We will show, in brief, how the argument works, as shown by
  Groth [\citeyear{Groth09}] and Gibson [\citeyear{Gibson22}].

  Assuming that the Prover has two vectors $\mathbf{x}, \mathbf{y} \in \mathbb{Z}_p^n$
  of size $n$, and a scalar $z$ such that $z = \langle \mathbf{x}, \mathbf{y} \rangle$
  \cite{Gibson22}. We construct three commitments:
  \begin{align}
    C_z = tH + zG \\
    C_x = rH + \sum_{i=1}^n x_i G_i \\
    C_y = sH + \sum_{i=1}^n y_i G_i
  \end{align}

  The algorithm will work in three steps:
  \begin{enumerate}
    \item \textbf{Commitment} The Prover creates four additional commitments:
      \begin{align}
        A_d = r_dH + \sum_{i=1}^n d_{xi}G_i \\
        B_d = s_dH + \sum_{i=1}^n d_{yi}G_i \\
        C_1 = t_1H + (\langle \mathbf{x}, \mathbf{d}_y \rangle + \langle \mathbf{y}, \mathbf{d}_x \rangle) \\
        C_0 = t_0H + (\langle \mathbf{d}_x, \mathbf{d}_y ) \rangle G
      \end{align}
      $A_d$ and $B_d$ are commitments for two nonce vectors $\mathbf{d}_x$ and
      $\mathbf{d}_y$ one for each of $\mathbf{x}$ and $\mathbf{y}$ \cite{Gibson22}.
      $C_1$ and $C_0$ are commitments to the inner product of the \emph{blinded}
      form of our vectors, this will return useful later in the response phase
      \cite{Gibson22}.
    \item \textbf{Challenge} The challenge phase simpy consists of the Verifier
      sending a scalar value $e$ as challenge \cite{Gibson22}.
    \item \textbf{Response} The Prover computes the following and send them to the
      Verifier:
      \begin{align}
        \mathbf{f}_x = e\mathbf{x} + \mathbf{d}_x \\
        \mathbf{f}_y = e\mathbf{y} + \mathbf{d}_y \\
        r_x = er + r_d \\
        s_y = es + s_d \\
        t_z = e^2t + et_1 + t_0
      \end{align}
      The verifier will then check that:
      \begin{align}
        eC_x + A_d = r_xH + \sum_{i=1}^n (\mathbf{f}_x)_iG_i\\
        eC_y + B_d = s_yH + \sum_{i=1}^n (\mathbf{f}_y)_iG_i
      \end{align}
      If the two sides of the equations are equal then we proven honest behaviour
      from the prover's side \cite{Gibson22}. Finally, we need to perform a third
      check to verify the correctness of the inner product:
      \begin{equation}
        \langle \mathbf{f}_x, \mathbf{f}_y \rangle = e^2z + e(\langle \mathbf{x}, \mathbf{d}_y \rangle + \langle \mathbf{y}, \mathbf{d}_x \rangle) + (\langle \mathbf{d}_x, \mathbf{d}_y \rangle )
      \end{equation}
  \end{enumerate}

  \subsubsection{Non-Interactability}

  As shown above, Bulletproofs require interaction between the Prover and the
  Verifier. We can render this protocol non-interactive through the use of a
  Fiat-Shamir Heuristic and generating the challenge using the public data of the
  proof.

  \section{Methodology}

  In this section we will discuss our implementation as well as the library used in
  this project.

  \subsection{Design}

  The program functions in three stages:
  \begin{enumerate}
    \item \textbf{Passport Encoding}
    \item \textbf{Proof Construction}
    \item \textbf{Proof Verification}
  \end{enumerate}

  In the first stage the program constructs a \texttt{Passport} object from
  string values. The \texttt{Passport} object has one variable for each field
  in the MRZ of a phisical passport. This data will later be used to construct the
  proof.

  The second stage consists in first computing the hash of the \texttt{Passport}
  object. This is achieved by converting all variables in the objects into arrays
  of bytes and concatenating together. The final array will be passed to a hash
  function. The hash function chosen for this project is \texttt{SHA3-256}, which
  was chosen due to the small size of the resulting hash and its resistance against
  attacks. Once the hash is computed, the proof is then constructed. The program
  begins by randomly selecting the generator vector $\mathbf{G}$ of size $n = 32$.
  Then, we create the Pedersen generator $F$ and the blinding value $B$. In the next
  step, we have to determine a public vector $\mathbf{b}$ that will be used in
  computing the inner product $c$ with our secret vector $\mathbf{a}$. We determine
  such vector randomly using the cryptographically secure random number generator
  (CSRNG) offered by the \texttt{rand} crate. For our secret vector $\mathbf{a}$,
  we use the hash value of the passport to fill in all entries. Next, the program
  initializes the \texttt{merlin} transcript that is used to render the protocol
  non-interactive, and the blinding value of the commitment $r$. Finally, we compute
  the inner product $c$ using our secret vector $\mathbf{a}$ and the public vector
  $\mathbf{b}$ and the commitment $C$ using the formula
  \begin{equation}
    C = \sum_{i=0}^{n-1} a_iG_i + rB + cF
  \end{equation}

  The last step in the proof creation is to invoke the \texttt{create} function
  from the \texttt{bulletproofs} crate to create the actual Bulletproofs proof
  and create the object \texttt{ProofEnv}, which contains all public values that
  will be used in the verification step.

  The verification step simply consists in creating the transcript for the Verifier
  calling the \texttt{verify} method of
  the \texttt{LinearProof} object and passing in the public values computed in the
  proof creation. If the proof succeeds then the program will display true, otherwise
  it will throw an \texttt{VerificationException}.

  \subsection{Tools Used}

  To implement our solution we used the Rust programming language as it supports
  the most modern and well-maintained Bulletproofs implementation. This library
  relies on Ristretto to generate the elliptic curve needed and Merlin,
  a transcript construction for ZKP that automates the Fiat-Shamir heuristic
  \cite{VYA18, VAL18}.

  During the initial stages of this project we consider other Bulletproofs
  implementations. One such library is the Bulletproofs library available for
  Haskell, which was originally planned for this project. However, we had to
  reconsider our choice due to the age of the library and the last published
  update of it. This library has not been maintained for the past 6 years making
  it outdated and thus unfit to be used for this project.

  \section{Future Work}

  In this section we will touch upon future work that could expand on this project.

  To start, we would like to expand on the current implementation and add the ability
  of, not only using passports, but also any form of eID that abides by ICAO Doc 9303
  and has a MRZ. This would improve on the usability of this techniques by allowing
  individuals that do not possess a passport to use it.

  Another expansion that in our opinion is vital for this project, is the addition
  of a TCP server and client to the program to showcase the functionality of the
  system better.

  Finally, another addition could be the ability of a user of entering passport
  data by reading the RFID chip on the ePassport. Originally this was one of the
  goals of the project, but it was later discarded due to difficulties in reading
  the chip as ePassports use sophisticated protections to avoid unauthorized reading
  of the data.

  \section{Conclusion}

  As we have discussed, this project is aimed at improving the security of sharing
  IDs over the internet, specifically to entities we might not fully trust with
  our PII by using ZKP protocols. We discussed background knowledge of MRTD such as
  passports and the information contained in their MRZ. We presented a background of
  ZKP in brief and explored Bulletproofs, as well as its component, more in depth.

  Finally, we outlined the work we carried out by presenting the design of our solution
  and how it functions. Later, we discussed the tools used to implement our program
  and future work necessary to improve upon our proposed solution.

  \bibliographystyle{ACM-Reference-Format}
  \bibliography{report}
\end{document}
